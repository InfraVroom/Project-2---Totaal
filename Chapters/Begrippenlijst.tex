\section{Verklarende Woordenlijst}
%\addcontentsline{toc}{section}{Verklarende Woordenlijst}
\printglossaries


\newglossaryentry{SLAM}
{
    name=\textit{SLAM},
    description={Simultaneous Localization And Mapping: Een methode om tegelijkertijd de locatie van een object te bepalen en een kaart van de omgeving te maken.}
}

\newglossaryentry{autonoom}
{
    name=\textit{autonoom},
    description={Zelfstandig of zonder menselijke tussenkomst.}
}

\newglossaryentry{Bluetooth}
{
    name=\textit{Bluetooth},
    description={Draadloze communicatietechnologie voor korte afstanden.}
}

\newglossaryentry{Smart-Car}
{
    name=\textit{Smart-Car},
    description={Type auto dat ontworpen is.}
}

\newglossaryentry{motorshield}
{
    name=\textit{motorshield},
    description={Printplaat met componenten die motoren aanstuurt.}
}

\newglossaryentry{shift-register}
{
    name=\textit{shift-register},
    description={Digitaal geheugen element voor seriële gegevensverwerking.}
}

\newglossaryentry{74HC595N}
{
    name=\textit{74HC595N},
    description={Een naam van een type shiftregister.}
}

\newglossaryentry{H-brug}
{
    name=\textit{H-brug},
    description={Elektronische schakeling voor het omkeren van stroomrichting.}
}

\newglossaryentry{microcontroller}
{
    name=\textit{microcontroller},
    description={Kleine computer op een chip.}
}

\newglossaryentry{register}
{
    name=\textit{register},
    description={tijdelijke geheugenopslag.}
}

\newglossaryentry{PWM}
{
    name=\textit{PWM},
    description={Pulserend signaal voor analoge regeling.}
}

\newglossaryentry{thermische}
{
    name=\textit{thermische beveiliging},
    description={beschermt apparaten tegen oververhitting en schade.}
}

\newglossaryentry{RF433MHZ}
{
    name=\textit{RF433MHZ},
    description={Draadloze communicatie op een frequentie van 433 MHz.}
}

\newglossaryentry{TFT-display}
{
    name=\textit{TFT-display},
    description={Dunne-film-transistor display voor grafische weergave.}
}

\newglossaryentry{ili9341}
{
    name=\textit{ili9341},
    description={controller IC voor aansturing van TFT-display.}
}

\newglossaryentry{ESP32 D1 MINI microcontroller}
{
    name=\textit{ESP32 D1 MINI microcontroller},
    description={Krachtige chip voor draadloze communicatie en complexe taken.}
}

\newglossaryentry{KY-023 joysticks}
{
    name=\textit{KY-023 joysticks},
    description={Analoge besturingselementen voor nauwkeurige bediening.}
}

\newglossaryentry{Transceiver}
{
    name=\textit{Transceiver},
    description={Apparaat voor verzending en ontvangst van draadloze signalen.}
}

\newglossaryentry{Tx en Rx}
{
    name=\textit{Tx en Rx},
    description={Pinnen die zorgen voor datacommunicatie door middel van UART.}
}

\newglossaryentry{WS2812B}
{
    name=\textit{WS2812B},
    description={Ledstrip waarbij elke individuele led afzonderlijk aangestuurd kan worden.}
}

\newglossaryentry{ESP8266}
{
    name=\textit{ESP8266},
    description={Krachtige en goedkope microcontroller met ingebouwde Wi-Fi-functionaliteit.}
}

\newglossaryentry{WLED}
{
    name=\textit{WLED},
    description={Open source LED-besturingssoftware}
}

\newglossaryentry{DC motoren}
{
    name=\textit{DC motoren},
    description={Elektromechanische apparaten die elektrische energie omzetten in mechanische energie.}
}
\newglossaryentry{IC}
{
    name=\textit{IC},
    description={Intergrated Circuit: Klein elektronisch apparaat die een groot aantal componenten en functies op één chip integreren.}
}

\newglossaryentry{underglow}
{
    name=\textit{underglow},
    description={Een soort verlichting die onder de auto is geïnstalleerd en de grond eronder verlicht.}
}

\newglossaryentry{HW-201}
{
    name=\textit{HW-201},
    description={Een met draaiweerstand kalibreerbare infrarood sensor.}
}

\newglossaryentry{HC-SR04}
{
    name=\textit{HC-SR04},
    description={Een naam van een type ultrasoon sensor.}
}

\newglossaryentry{transducer}
{
    name=\textit{transducer},
    description={Een apparaat dat signalen omzet van de ene vorm van energie naar een andere.}
}

\newglossaryentry{BNO055}
{
    name=\textit{BNO055},
    description={Een naam van een elektronisch component dat informatie kan verzamelen over de oriëntatie, positie en beweging van een object in de ruimte.}
}

\newglossaryentry{L293D}
{
    name=\textit{L293D},
    description={Een naam van een type H-brug.}
}

\newglossaryentry{Arduino Mega}
{
    name=\textit{Arduino Mega},
    description={De naam van de hoofd microcontroller op de auto.}
}