\section{Inleiding}
Begin semester twee is er een project gestart. Voor dit project is er, door de opdrachtgever Dhiradj Djairam, aan meerdere groepen gevraagd een zo geheten \gls{Smart-Car} te ontwerpen en te realiseren. De groep Infra Vroom is één van deze groepen.
De \gls{Smart-Car} moet van de opdrachtgever autonoom een doel kunnen detecteren en er autonoom naartoe kunnen rijden terwijl het obstakels vermijdt. Daarnaast moet het ook op afstand bedienbaar zijn met een \gls{Bluetooth}-app.
Bij het ontwerpen van het gevraagde product is het in het belang van de opdrachtgever dat de groepsleden hun programmeervaardigheden en kennis uitbreiden. Dit resulteert namelijk in het beste eindproduct. 

Het doel van dit rapport is antwoord geven op de vraag: "Welke stappen hebben geleid tot de huidige versie van de smart-car?" 
Dit zal worden beantwoord door middel van deelvragen, waarin wordt behandeld uit welke onderdelen de \gls{Smart-Car} bestaat, wat deze doen en hoe het algoritme voor het autonoom rijden is ontwikkeld. Om deze vragen te beantwoorden is de groep Infra Vroom begonnen met de eisen van de opdrachtgever doorlezen en het ontwerpen van de \gls{Smart-Car}. Daarnaast is bij elke deelvraag per onderdeel de deelvraag beantwoord. 

Om de hoofdvraag te beantwoorden zal allereerst het analyse proces en het programma van eisen doorlopen worden. Vervolgens zullen de ontwerpconcepten kort besproken worden. Verder zullen de deelvragen beantwoord worden. Deze zijn onderverdeeld in meerdere subsecties en zullen behandelen, uit welke onderdelen de auto bestaat, welke hardware aanpassingen er zijn gemaakt sinds assessment 1, hoe het algoritme tot stand is gekomen en hoe de documentatie is bijgehouden. Ten slotte volgt de conclusie met daarna nog de aanbevelingen. Onderaan het rapport staan ook nog de volledige code en referenties.