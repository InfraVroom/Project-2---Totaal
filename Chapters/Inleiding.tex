\section{Inleiding}
Begin semester twee is er een project gestart. Voor dit project is er, door de opdrachtgever Dhiradj Djairam, aan meerdere groepen gevraagd een zo geheten \gls{Smart-Car} te ontwerpen en te realiseren. De groep Infra Vroom is één van deze groepen.
De \gls{Smart-Car} moet van de opdrachtgever autonoom kunnen rijden en obstakels kunnen ontwijken. Daarnaast moet het ook op afstand bedienbaar zijn met een \gls{Bluetooth}-app.
Bij het ontwerpen van het gevraagde product is het in het belang van de opdrachtgever dat de groepsleden hun programmeervaardigheden en kennis uitbreiden. Dit resulteert namelijk in het beste eindproduct. 

Het doel van dit rapport is antwoord geven op de vraag: welke stappen zijn er genomen bij het ontwerpen en realiseren van het product voor assessment 1? 
Dit zal worden beantwoord door middel van deelvragen, waarin wordt behandeld welke onderdelen de \gls{Smart-Car} uit bestaat, wat deze doen en hoe ze getest zijn. Om deze vragen te beantwoorden is de groep Infra Vroom begonnen met de eisen van de opdrachtgever doorlezen en het ontwerpen van de \gls{Smart-Car}. Daarnaast is er bij elke deelvraag per onderdeel de deelvraag gesteld en beantwoord. 

Om de hoofdvraag te beantwoorden zal allereerst het analyse proces en het programma van eisen doorlopen worden. Vervolgens zullen de ontwerpconcepten kort besproken worden. Verder zullen de deelvragen beantwoord worden. Deze zijn onderverdeeld in meerdere subsecties en zullen behandelen, uit welke onderdelen de auto bestaat, hoe de tussenproducten getest zijn en hoe de documentatie is bijgehouden. Ten slotte volgt de conclusie met daarna nog de aanbevelingen. Onderaan het rapport staan ook nog de volledige code en referenties.  