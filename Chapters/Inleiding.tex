\section{Inleiding}
Begin semester 2 is er een project gestart. Voor dit project is gevraagd een zo geheten Smart Car te ontwerpen en realiseren. Daarom is de groep Infra Vroom begonnen met de eisen van de opdrachtgever doorlezen en het ontwerpen van de \gls{Smart-Car}. 

Het doel van dit ontwerprapport is het verduidelijken van de tot nu toe uitgevoerde stappen en de gedachtegang hierachter. Daarnaast zullen de belangrijkste acties en onderdelen uitgelegd worden. 

Dit zal gedaan worden door de hoofdvraag te beantwoorden. De hoofdvraag luidt: "Welke stappen zijn er genomen bij het ontwerpen en realiseren van het product voor assessment 1?". Allereerst zal het analyse proces en het programma van eisen doorlopen worden. Vervolgens zullen de ontwerpconcepten kort besproken worden. 

 Verder zal de hoofdvraag worden uitgewerkt met behulp van deelvragen. Deze zijn onderverdeeld in meerdere subsecties en zullen behandelen, uit welke onderdelen de auto bestaat, hoe de tussenproducten getest zijn en hoe de documentatie is bijgehouden. Ten slotte volgt de conclusie met daarna nog de aanbevelingen. Onderaan het rapport staan ook nog de volledige code en referenties.  
