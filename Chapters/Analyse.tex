\section{Analyse}
In opdracht van de heer Dhiradj Djairam is er een casus gegeven om een \gls{Smart-Car} te ontwerpen die \gls{autonoom} kan rijden, in staat is obstakels te detecteren en te vermijden en die op afstand bestuurd kan worden via \gls{Bluetooth}. 

Het ontwerp dient dus te voorzien in de behoefte naar een slimme \gls{Smart-Car} met autonome rijfuncties, obstakeldetectie- en vermijdingstechnologieën. Daarnaast dient het ontwerp te beschikken over \gls{Bluetooth}-functionaliteit om op afstand besturen mogelijk te maken. Door deze kenmerken te combineren, biedt het ontwerp een oplossing voor de casus van het ontwerpen van een \gls{Smart-Car} die \gls{autonoom} kan rijden en obstakels kan detecteren en vermijden.

Dit zal moeten worden gerealiseerd door gebruik van verschillende sensoren en modules. Zo zullen er ultrasoon en infrarood sensoren gebruikt worden om het autonoom rijden mogelijk te maken en een \gls{Bluetooth} module om het op afstand bedienen van de \gls{Smart-Car} te verwezenlijken. 

%en een fysieke controller voor een alternatieve besturingsmethode. Door het gebruik van \gls{SLAM}-algoritmen kan de auto beter \gls{autonoom} rijden en obstakels detecteren, vermijden of vinden.

 %Dit betekent dat de \gls{Smart-Car} in staat is om op zichzelf te rijden, obstakels te detecteren en te vermijden. De \gls{Bluetooth}-functionaliteit maakt op afstand besturen mogelijk. 
 
 %en de fysieke controller biedt een alternatieve besturingsmethode. Het gebruik van \gls{SLAM}-algoritmen zorgt ervoor dat de \gls{Smart-Car} beter in staat is om \gls{autonoom} te rijden en obstakels te detecteren, vermijden of vinden. 