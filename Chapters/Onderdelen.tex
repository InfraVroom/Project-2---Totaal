\section{Uit welke onderdelen bestaat de zelfrijdende auto en waarvoor worden deze gebruikt?}
\subsection{Sensoren}
\subsection{Motoren}
Voor het aansturen van de motoren wordt gebruik gemaakt van een motorshield, bestaande uit twee belangrijke componenten: een 74HC595N shift register en twee L293D H-brug motor drivers. Het 74HC595N shift register is een seriële-in, parallel-out shift register met uitgangslatches, dat 8-bits extra digitale uitgangen creëert zonder extra pinnen op de microcontroller te gebruiken. Om gegevens naar de chip te sturen, dient men seriële gegevens in te voeren op de SER-pin en de klok-ingang SRCLK pulsen te geven. Bij iedere puls wordt het volgende seriële bit overgezet naar het register. Na overdracht van alle 8 bits kan men de inhoud van het register op de parallelle uitgangen zien. Om de inhoud van het register op de parallelle uitgangen te laten verschijnen, dient men een latch-puls te geven door de RCLK-pin te activeren.\\\\
De uitgangslatches van het shift register sturen een positief of negatief signaal naar de twee L293D H-brug motor drivers, waarmee de vier motoren worden aangestuurd. Met de L293D kan de richting en snelheid van een DC-motor worden geregeld met behulp van een microcontroller of andere digitale circuits. Het IC beschikt over vier digitale ingangen (twee voor elke motor), waarmee de motor in de gewenste richting kan worden gestuurd. Om de motor aan en uit te zetten, beschikt de L293D tevens over één Enable-ingang per motor, die kan worden aangesloten op een PWM-signaal om de motorsnelheid te regelen.\\\\
Het is van groot belang om de stroomsterkte van de motor en het voltage van de voeding goed te controleren en af te stemmen op de specificaties van de L293D. Dit omdat het IC een maximale stroom van 600 mA per kanaal kan leveren en is uitgerust met ingebouwde beveiligingsfuncties, zoals thermische uitschakeling en bescherming tegen kortsluiting, wat het veilig en betrouwbaar maakt om te gebruiken in diverse toepassingen.\\\\
Om een DC-motor in één richting te laten draaien, moeten de logische waarden op IN1 en IN2 voor motor 1 respectievelijk op HIGH en LOW worden gezet. Wanneer de Enable-ingang van de motor hoog is, wordt de motor ingeschakeld, en wanneer deze laag is, wordt de motor uitgeschakeld. Bovendien hebben we na uitgebreid onderzoek vastgesteld dat de maximale PWM-waarde voor de motoren niet hoger dan 220 mag zijn, omdat anders de motoren kunnen doorbranden. Dankzij het gebruik van het motorshield met de 74HC595N shift register en L293D H-brug motor driver kunnen de motoren op een gecontroleerde en veilige manier worden aangestuurd.

\subsection{Matrix}
\subsection{Modules}
\subsubsection{Controller}
Tijdens de ontwikkeling van ons project hebben wij een controller ontworpen om een auto aan te sturen met behulp van RF433MHZ-chips. Verschillende componenten werden hiervoor gebruikt, waaronder een TFT-display met een ili9341 IC, twee KY-023 joysticks en een ESP32 D1 MINI microcontroller.\\\\
De ESP32 D1 MINI microcontroller is een krachtige microcontroller die in staat is om draadloze verbindingen te maken en complexe taken uit te voeren. Met zijn dubbele kern en 240 MHz rekenkracht was hij bij uitstek geschikt voor het verwerken van de gegevens van de twee KY-023 joysticks. De ingebouwde WiFi- en Bluetooth-functionaliteit maakte het bovendien mogelijk om draadloos verbinding te maken met andere apparaten, wat tijdens de ontwikkeling van het project handig was.\\\\
Het TFT-display met de ili9341 IC zorgde voor een snelle en efficiënte verwerking van de grafische weergave op het scherm. Dit stelde ons in staat om snel en vloeiend de informatie weer te geven die we nodig hadden om de auto te besturen.\\\\
De twee KY-023 joysticks waren aangesloten op de ESP32 D1 MINI en na het uitlezen van de waarden, werden deze gekalibreerd en omgezet in een serieel formaat dat verzonden kon worden via de RF433MHZ-transceiver. De combinatie van RF433MHZ en de joysticks zorgde voor een betrouwbare en responsieve verbinding tussen de controller en de auto, waardoor we de auto met precisie konden besturen in elke richting.\\\\
Over het algemeen waren het TFT-display met de ili9341 IC en de krachtige ESP32 D1 MINI microcontroller cruciale componenten in het ontwerp van onze controller. Ze zorgden voor een vloeiende en responsieve gebruikerservaring en gaven ons de flexibiliteit en rekenkracht die we nodig hadden om de auto nauwkeurig te kunnen besturen.\\\\
Daarnaast is het gebruik van de KY-023 joysticks ook een cruciaal component in het ontwerp van onze controller. De joysticks maakten het mogelijk om de auto in elke richting te bewegen en te sturen, waardoor we nauwkeurige controle hadden over de auto. Dankzij de combinatie van RF433MHZ en de joysticks konden we de auto met precisie besturen en de betrouwbare en responsieve verbinding tussen de controller en de auto zorgde voor een vloeiende gebruikerservaring.
\subsection{Behuizing}