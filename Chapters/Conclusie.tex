\section{Conclusie}
De hoofdvraag luidde: "Welke stappen zijn er genomen bij het ontwerpen en realiseren van het product voor assessment 1?"
Aan de hand van deelvragen, is hier antwoord op gegeven. 
Zo beschikt de \gls{Smart-Car} van een behuizing met 3 verschillende sensoren. De infrarood sensoren worden gebruikt om te bepalen of er een obstakel in de weg staat, de ultrasoon sensoren worden gebruikt om de afstand tot een object te bepalen, de 9 axis sensor wordt gebruikt om de \gls{Smart-Car} recht te houden. Ook heeft de \gls{Smart-Car} vier motoren om zich mee te verplaatsen en een servo om de dode hoek van de voorste ultrasoon sensor te verkleinen. Verder zit er nog een matrix op de achterkant van de auto die dient om de oriëntatie richting weer te geven. Tenslotte is er ook een \gls{Bluetooth}-module gemonteerd op de auto, waarmee op afstand besturen mogelijk wordt gemaakt. 

Al deze componenten zijn getest door veel onderzoek te doen, uit te proberen of het werkt, fouten verbeteren en herhaal. Alles is zo uitgewerkt tot het één geheel en goed werkende \gls{Smart-Car} vormde. 

 Gedurende dit hele proces is ook een documentatie bijgehouden in Github en in de gezamenlijke Onedrive. 