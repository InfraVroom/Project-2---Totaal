\section{Conclusie}
De hoofdvraag luidde: "Welke stappen hebben geleid tot de huidige versie van de smart-car?"
Aan de hand van deelvragen, is hier antwoord op gegeven. 

\begin{comment}
Zo beschikt de \gls{Smart-Car} van een behuizing met 3 verschillende sensoren. De infrarood sensoren worden gebruikt om te bepalen of er een obstakel in de weg staat, de ultrasoon sensoren worden gebruikt om de afstand tot een object te bepalen, de negen-axis sensor wordt gebruikt om de \gls{Smart-Car} recht te houden. Ook heeft de \gls{Smart-Car} vier motoren om zich mee te verplaatsen en een servo om de dode hoek van de voorste ultrasoon sensor te verkleinen. Verder zit er nog een matrix op de achterkant van de auto die dient om de oriëntatie richting weer te geven. Tenslotte is er ook een \gls{Bluetooth}-module gemonteerd op de auto, waarmee op afstand besturen mogelijk wordt gemaakt. 
\end{comment}

De \gls{Smart-Car} heeft een behuizing met 3 sensoren: infrarood sensoren voor obstakeldetectie, ultrasoon sensoren voor afstandsmeting en een negen-axis sensor die in dit geval niet wordt gebruikt door een te complexe code. Daarnaast heeft hij vier motoren voor voortbeweging, een servo om de dode hoek van de voorste ultrasoon sensor te verminderen, een matrix voor oriëntatie en een Bluetooth-module voor het op afstand besturen van de \gls{Smart-Car}.De controller wordt niet gebruikt, omdat de componenten defect waren geraakt en wegens tijdgebrek.

Er zijn drie verschillende algoritmes geschreven. Het eerste algoritme wordt gebruikt voor het ontwijken van obstakels. Het tweede algoritme bouwt voort op het eerste algoritme. Alleen is de periscoop op de middelste servo van de \gls{Smart-Car} gemonteerd, waarbij de servo 180 graden draait om vervolgens een laserzuil te kunnen detecteren. Het derde en laatste algoritme bouwt voort op de twee vorige algoritmes. In dit geval gaat de \gls{Smart-Car} 180 graden draaien als de servo niets heeft gedetecteerd.

Al deze componenten, algoritmes en tussenproducten zijn getest door veel onderzoek te doen, uit te proberen of het werkt, fouten verbeteren en herhaal. Alles is zo uitgewerkt tot het één geheel en goed werkende \gls{Smart-Car} vormde. 

 Gedurende dit hele proces is ook een documentatie bijgehouden in Github en in de gezamenlijke Onedrive. 