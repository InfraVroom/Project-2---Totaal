\phantomsection
\section*{Samenvatting}\addcontentsline{toc}{section}{Samenvatting}
Het project microcontroller is begin semester twee gestart. Hierbij is in opdracht van de opdrachtgever Dhiradj Djairam de groep Infra Vroom aan de slag gegaan met het maken van een \gls{Smart-Car}. Het doel was een \gls{Smart-Car} ontwerpen en realiseren die in staat is obstakels te detecteren en ontwijken en op afstand bedienbaar is met een \gls{Bluetooth}-app. Bij dit proces is het verbeteren van de ontwerp- en programmeervaardigheden van de groepsleden voor het beste resultaat en dus voor de opdrachtgever van belang.

Dit rapport geeft antwoord op de vraag: welke stappen zijn er genomen bij het ontwerpen en realiseren van het product voor assessment 1? Dit wordt beantwoord door middel van deelvragen, die zijn uitgewerkt.

 De \gls{Smart-Car} bestaat uit verschillende onderdelen, waaronder de behuizing, banden, sensoren en actuatoren. De assemblage van de \gls{Smart-Car} is bij de Kick-off van het project gebeurd. Hierbij zijn de gegeven onderdelen van de behuizing, met spacers op elkaar vastgeschroefd. De sensoren die op de \gls{Smart-Car} zitten, zijn: infrarood\cite{IR-datasheet} sensoren , ultrasoon sensoren en één negen axis sensor. Er zitten in totaal acht infrarood sensoren op de \gls{Smart-Car}, twee op elke hoek, en worden gebruikt om objecten te detecteren. De \gls{Smart-Car} bevat drie ultrasoon sensoren, één op elke kant. De sensor op de voorkant is bevestigd op een servomotor zodat deze heen en weer kan bewegen, en zo de dodehoek van de sensoren minimaal is. De ultrasoon sensoren worden gebruikt om afstanden te meten. De gegevens van de sensoren kunnen in het algoritme worden gebruikt voor obstakeldetectie en voor de autonome aansturing van de \gls{Smart-Car} door zijn omgeving. 
 
 De Actuatoren zijn de motoren en de servo. De motoren worden aangestuurd door een \gls{motorshield} dat bestaat uit een \gls{shift-register} en \gls{H-brug} motor drivers\cite{h-brug}. Dit wordt gedaan door acht bits te sturen via de motorshield, die de motoren aansturen om vooruit of achteruit te gaan en met welke snelheid. De negen axis sensor meet de oriëntatie van de \gls{Smart-Car} en wordt momenteel alleen gebruikt om de auto rechtdoor te laten rijden. 

 Om de infrarood sensoren\cite{IR-datasheet} te testen, moeten ze worden aangesloten op het \gls{motorshield} en moet de digitale pin op de Arduino Mega 2560\cite{ArduinoMEGA} worden gedefinieerd als de invoerpin. Ultrasone sensoren worden getest door de VCC- en GND-pinnen aan te sluiten op het \gls{motorshield}, en de trigger- en echo-pinnen op een digitale pin van de Arduino Mega 2560\cite{ArduinoMEGA}. De echo-pin wordt gebruikt om de afstand te bepalen door de tijd te meten die nodig is voor het ultrasone geluid om terug te kaatsen en hiermee vervolgens de berekening "afstand = de gemeette tijd × de geluidssnelheid" uit te voeren. De motordriver\cite{h-brug} wordt aangestuurd door de shiftregisters\cite{shiftregister}, welke een 8-bits signaal decodeert. Hiervoor is een Excel-script geschreven dat testen heel makkelijk maakte.

 Gedurende het gehele ontwerp- en realisatieproces van de \gls{Smart-Car} is een documentatie bijgehouden in Github en in de gezamenlijke Onedrive. Ook is in Github de planning bijgehouden met alle taken voor het maken van de tussenproducten en documenten. 

De hoofdvraag luidde: "Welke stappen zijn er genomen bij het ontwerpen en realiseren van het product voor assessment 1?" De genomen stappen zijn: 
\begin{enumerate}
    \item Het analyseerproces en het bekijken van het programma van eisen.
    \item De sensoren en actuatoren en overige onderdelen van de \gls{Smart-Car} uitwerken.
    \item De tussenproducten testen en ervoor zorgen dat deze werkten zoals gewenst.
    \item De documentatie bijhouden met behulp van Github en een gezamenlijke Onedrive. 
\end{enumerate}
