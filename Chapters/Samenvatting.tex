\phantomsection
\section*{Samenvatting}\addcontentsline{toc}{section}{Samenvatting}
Het project microcontroller is begin semester twee gestart. Hierbij is in opdracht van de opdrachtgever Dhiradj Djairam de groep Infra Vroom aan de slag gegaan met het maken van een \gls{Smart-Car}. Het doel was een \gls{Smart-Car} ontwerpen en realiseren die in staat is een doel te vinden, obstakels te detecteren en ontwijken, en op afstand bedienbaar is met een \gls{Bluetooth}-app. Bij dit proces is het verbeteren van de ontwerp- en programmeervaardigheden van de groepsleden voor het beste resultaat en dus voor de opdrachtgever van belang.

Dit rapport geeft antwoord op de vraag: "Welke stappen hebben geleid tot de huidige versie van de smart-car?". Dit wordt beantwoord door middel van deelvragen, die zijn uitgewerkt.

De \gls{Smart-Car} bestaat uit verschillende onderdelen, waaronder de behuizing, banden, sensoren en actuatoren. De assemblage van de \gls{Smart-Car} is bij de Kick-off van het project gebeurd met achteraf nog wat toevoegingen. De sensoren die op de \gls{Smart-Car} zitten, zijn: infrarood\cite{IR-datasheet}, ultrasoon en één negen-axis sensor. Er zitten in totaal acht infrarood sensoren op de \gls{Smart-Car}, om objecten te detecteren. De \gls{Smart-Car} bevat drie ultrasoon sensoren. De ultrasoon sensoren worden gebruikt om afstanden te meten. De gegevens van de sensoren kunnen in het algoritme worden gebruikt voor obstakeldetectie en voor de autonome aansturing van de \gls{Smart-Car} om objecten te ontwijken. De Actuatoren zijn de motoren en de servo. De motoren worden aangestuurd door een \gls{motorshield} dat bestaat uit een \gls{shift-register} en \gls{H-brug} motor drivers\cite{h-brug}. Dit wordt gedaan met een pwm-signaal, die de motoren aanstuurt om vooruit of achteruit te gaan en met welke snelheid. De negen-axis sensor wordt vanwege het ontstaan van te complexe code bij gebruik niet meer gebruikt.  

Om de infrarood- en ultrasone sensoren te testen, moeten ze worden aangesloten op het \gls{motorshield} en Arduino Mega 2560\cite{ArduinoMEGA}. De echo-pin meet de afstand op basis van de terugkaatstijd van ultrasone geluidsgolven. Het aansturen van de motordriver\cite{h-brug} gebeurt met een \gls{shift-register}. Er is ook een Excel-script geschreven om het testen van de motoren te vereenvoudigen.

\begin{comment}
Om de infrarood sensoren\cite{IR-datasheet} te testen, moeten ze worden aangesloten op het \gls{motorshield} en moet de digitale pin op de Arduino Mega 2560\cite{ArduinoMEGA} worden gedefinieerd als de invoerpin. Ultrasone sensoren worden getest door de VCC- en GND-pinnen aan te sluiten op het \gls{motorshield}, en de trigger- en echo-pinnen op een digitale pin van de Arduino Mega 2560\cite{ArduinoMEGA}. De echo-pin wordt gebruikt om de afstand te bepalen door de tijd te meten die nodig is voor het ultrasone geluid om terug te kaatsen en hiermee vervolgens de berekening "afstand = de gemeette tijd × de geluidssnelheid" uit te voeren. De motordriver\cite{h-brug} wordt aangestuurd door de shiftregisters\cite{shiftregister}, welke een 8-bits signaal decodeert. Hiervoor is een Excel-script geschreven dat testen heel makkelijk maakte.
\end{comment}

Er zijn drie verschillende algoritmes geschreven. Het eerste algoritme is het basis zelf rijd algoritme dat zorgt voor het ontwijken van obstakels. Het tweede algoritme heeft een toevoeging op het eerste algoritme. Hierbij is de periscoop op de middelste servo van de \gls{Smart-Car} gemonteerd, waarbij de \gls{Smart-Car} begint met het scannen van de omgeving op zoek naar het laserzuil. Dit doet hij door de servo maximaal 180 graden te draaien tot het, het laserzuil detecteert. Het derde en laatste algoritme bouwt voort op de twee vorige algoritmes. In dit geval zal de \gls{Smart-Car} wederom starten met het scannen van de omgeving, echter draait nu de gehele \gls{Smart-Car} 180 graden in plaats van de servo.

Gedurende het gehele ontwerp- en realisatieproces van de \gls{Smart-Car} is een documentatie bijgehouden in Github en in de gezamenlijke Onedrive. Ook is in Github de planning bijgehouden met alle taken voor het maken van de tussenproducten en documenten. 

De hoofdvraag luidde: "Welke stappen hebben geleid tot de huidige versie van de smart-car?" De genomen stappen zijn: 
\begin{enumerate}
    \item Het analyseerproces en het bekijken van het programma van eisen.
    \item De sensoren en actuatoren en overige onderdelen van de \gls{Smart-Car} uitwerken.
    \item De tussenproducten / verschillende versies van de algoritmes testen en ervoor zorgen dat deze werkten zoals gewenst.
    \item De documentatie bijhouden met behulp van Github en een gezamenlijke Onedrive. 
\end{enumerate}
