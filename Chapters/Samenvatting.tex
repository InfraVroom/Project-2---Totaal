\phantomsection
\section*{Samenvatting}\addcontentsline{toc}{section}{Samenvatting}
In het tweede semester is de groep Infra Vroom begonnen aan het project microcontroller, onder opdracht van de opdrachtgever "Dhiradj Djairam". Voor dit project is gevraagd een \gls{Smart-Car} te ontwerpen en realiseren. 

Hiertoe was de eerste stap het doornemen van de eisen van de opdrachtgever. In dit rapport worden alle stappen doorlopen die genomen zijn om tot het huidige product te komen.  

De hoofdvraag is: "Welke stappen zijn er genomen bij het ontwerpen en realiseren van het product voor assessment 1?". Hiertoe zal het analyse proces en het programma van eisen doorlopen worden. Daarna worden de ontwerpconcepten kort besproken worden. Verder is de hoofdvraag uitgewerkt met behulp van deelvragen, welke zullen behandelen, uit welke onderdelen de auto bestaat, hoe de tussenproducten getest zijn en hoe de documentatie is bijgehouden. Ten slotte volgt de conclusie met daarna nog de aanbevelingen. Onderaan het rapport staan ook nog de volledige code en referenties.  

Uit welke onderdelen bestaat de \gls{Smart-Car} en waarvoor worden deze gebruikt?
De \gls{Smart-Car} beschikt over een behuizing met 3 verschillende sensoren. De infrarood sensoren worden gebruikt om te bepalen of er een blokkade is, de ultrasoon sensoren om de afstand tot een object te bepalen en de 9 axis sensor wordt gebruikt om de \gls{Smart-Car} recht te houden. De \gls{Smart-Car} heeft ook vier motoren om zich mee voort te bewegen en een servo om de dode hoek van de voorste ultrasoon sensor te verkleinen. Verder zit er nog een matrix op de achterkant van de auto gebruikt wordt om de oriëntatie richting weer te geven. Tenslotte is er ook een \gls{Bluetooth}-module gemonteerd op de auto, waarmee op afstand besturen mogelijk wordt gemaakt. 

Hoe zijn de tussenproducten getest?
Alle componenten en tussen producten zijn getest door veel onderzoek te doen, uit te proberen of het werkt, de eventuele fouten te verbeteren en dit proces te herhalen tot het werkt zoals gewenst. Alles is zo uitgewerkt tot het één geheel vorm en het product een goed werkende \gls{Smart-Car} is. 

Hoe is de documentatie van het proces bijgehouden?
 Gedurende dit hele proces van het ontwerpen van de \gls{Smart-Car} is een documentatie bijgehouden in Github en in de gezamenlijke Onedrive. 