\section{Programma van eisen}

Het hoofddoel voor de demonstratie van assessment drie is om naast alle voorheen al gestelde eisen, autonoom naar de laser zuil te rijden zonder obstakels te raken. De volgende aspecten zijn belangrijk:
\begin{enumerate}
    \item Functionaliteit: De \gls{Smart-Car} moet autonoom kunnen navigeren, obstakels detecteren en vermijden.
    \item Volger: De \gls{Smart-Car} moet in staat zijn om een andere \gls{Smart-Car} te volgen met behulp van de periscoop en het laser signaal op de te volgen \gls{Smart-Car}.
    \item Reactievermogen: De \gls{Smart-Car} moet snel kunnen reageren op externe prikkels, zoals bewegingen van andere \gls{Smart-Car}'s, om botsingen te voorkomen.
    \item Samenwerking en competenties: Het project vereist effectieve samenwerking tussen teamleden met verschillende expertises, zoals sensortechnologie, algoritmeontwikkeling en mechanisch ontwerp.
    \item Competentie-ontwikkeling: Het project biedt de teamleden de mogelijkheid om waardevolle ervaring op te doen in robotica, sensortechnologie, algoritmeontwikkeling en mechanisch ontwerp, en om hun vaardigheden te verbeteren en te leren werken in een teamomgeving.
\end{enumerate}
