\section{Aanbevelingen}
\begin{comment}
Er zijn nog veel dingen te verbeteren aan de \gls{Smart-Car}. Zo is de functionaliteit van de fysieke controller en het algoritme nog niet optimaal en kan de bekabeling nog veel beter. 
Aanbevolen wordt daarom, om in de toekomst verder te gaan met het ontwikkelen van de fysieke controller. Daarnaast is het autonome gedeelte van de \gls{Smart-Car} doormiddel van een algoritme gemaakt, die nog erg weinig omvat. Dit kan nog veel beter en verder uitgewerkt worden, onder andere door verdere toepassing van \gls{SLAM} met behulp van de 9 axis sensor. Verder kan de bekabeling heel wat netter en overzichtelijker gemaakt worden. Met behulp van speciaal ontworpen printplaatjes kan ook het aantal draadjes en de lengtes hiervan verkleind worden. 
\end{comment}

Er zijn nog een aantal dingen te verbeteren aan de \gls{Smart-Car}. Ten eerste moet de Smart-car worden opgeschoond, zoals een goede cable-management en eventuele veranderingen aan de behuizing voor fysieke knoppen, zodat tussen verschillende standen kan worden gewisseld. Ten tweede moet het uitlezen van de periscoop worden verbeterd. Tot nu toe kunnen er geen periscoop-waardes worden ingelezen, omdat de periscoop geen verschil in waardes meet. Ten slotte zou het gebruik van een snellere processor dan de Arduino Mega 2560 van pas kunnen komen, zodat er de mogelijkheid is om als projectgroep een keuze te kunnen maken tussen eventuele betere algoritmes. De ervaring leert dat de Arduino Mega 2560 te traag is voor algoritmes die meer rekenkracht vereisen.