\section{Ontwerpconcepten}
Aan het begin van het ontwerptraject zijn er concepten uitgedacht. Deze concepten zijn gemaakt om de ideeën te structureren, zodat aan het eind van het project, de \gls{Smart-Car} de gestelde doelen kan uitvoeren. 

Tijdens het opstellen van de concepten is gebrainstormd. Uit dit brainstormen kwamen wat ideeën. De eerste paar ideeën waren visueel gericht en hadden geen effect op de prestatie. Zo is er een \gls{underglow} gemonteerd en een spoiler ontworpen. 
Na de visuele ideeën werd nagedacht over prestatiegerichte toevoegingen. \gls{SLAM} toepassen met behulp van een 9 axis sensor was daar een van. Door \gls{SLAM} toe te passen kan een veel beter algoritme geschreven worden. Er is ook nagedacht over het maken van een fysieke controller. 
Verder is er natuurlijk nagedacht over de implementatie van de andere eisen en componenten die al standaard gegeven zijn. 

De zelfrijdende \gls{Smart-Car} zal in staat zijn om objecten te detecteren en te ontwijken of te volgen door middel van sensordata verwerkt in een algoritme dat keuzes kan maken. De \gls{Smart-Car} zal zelfstandig kunnen rijden van punt A naar punt B, waarbij punt B een baken is dat het moet vinden, terwijl het eventuele obstakels vermijdt. 
Ook zal de \gls{Smart-Car} in staat zijn bestuurd te worden door middel van \gls{Bluetooth} met een app op een telefoon. Deze \gls{Bluetooth}-app zal in de toekomst dus ook kunnen worden vervangen door een fysieke controller.
Het eindresultaat van het project is een \gls{Smart-Car}, welke in staat is: \gls{autonoom} blokkades te detecteren en te omzeilen, objecten te volgen en/of ervan weg te rijden en aangestuurd te worden via \gls{Bluetooth} en een fysieke controller. 
